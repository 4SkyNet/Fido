% Intro text here; a sentence or two
Intro text.

\subsection{Q-Learning}

Reinforcement learning seeks to find the optimal action to be undertaken for a given state through trial and error. In the context of Fido, an action could be the playing of a note or driving straight forward, while the state could the amount of light detected by the robot or how near the robot is to another object. Once an action is performed, a reward and a new state are given back to the reinforcement learning algorithm.  As actions are performed over time a reinforcement learning algorithm sharpens its ability to receive reward.

$Q$-Learning \cite{watkins} is a popular reinforcement learning algorithm that works by learning an action-value function $Q$ that takes an state-action pair as an input and outputs an expected utility value of performing that action in that state. This utility value is know as the $Q$-value. The $Q$-value is a combination of immediate reward and expected future reward. Every reward iteration, the $Q$-value of an state-action pair is updated as such:

\begin{equation}
	Q(s, a) := Q(s, a)(1 - \alpha) + \alpha(R + \gamma \max Q(s_{t+1}, a))
	\,,
\end{equation}

Where $a$ is the action carried out, $s$ is the initial state, $R$ is the reward received, and $s_{t+1}$ is the new state. $\alpha$ is the learning rate of the algorithm. The learning rate determines the rate of convergence by diminishing or amplifying the changes made to the $Q$-value each reward iteration. $\gamma$ is the devaluation factor, which determines the weight given to future rewards. A devaluation factor approaching $\gamma=0$ will force the algorithm to only value immediate reward, while a devaluation factor approaching $\gamma=1$ will make it focused on high long term reward.

The $Q$-Learning algorithm had to be immediately modified in two ways to make it suitable for Fido. Its scalability had to be improved, and it had to be able to work in continuous state-action spaces. 

$Q$-learning in its simplest form uses a table to model the $Q$ function, storing past state-action pairs and each pair's respective $Q$-value.  However, this strategy lacks scalability. In large state-action spaces, such as those Fido will have to perform in, the amount of data and computation needed to maintain such a table renders such a strategy impractical. Since Fido will be working in large state-action space, it is necessary to use a function approximator to model $Q$. A feed-forward neural networks were chosen for this task for their ability to model non-linear functions, lightweight computational footprint, high trainablility.

Conventional $Q$-Learning is discrete. No relation is made between states or actions, and every action for each state must be performed individually repeatedly in a noisy feedback system to determine its $Q$-value. However, Fido will work in a large, continuous state-action spaces where relations made between (state-action, $Q$-value) pairs can drastically reduce the number of reward iterations needed for convergence. An example of a task that would benefit from continuity is teaching Fido to adjust the speed of its motors based on the intensity of light that the robot detects. There is an obvious gradient relation between the (state-action, $Q$-value) pairs in this task and with a limited number of Q-values known, it is possible to correctly model $Q$.

\subsection{Wire-Fitted Q-Learning}

To accommodate continuous action-spaces, we coupled a wire-fitted moving least squares interpolator with our feed-forward neural network as described in \cite{gaskett}. 

Feed-forward neural networks can generalize between states in Q-Learning problems with discrete actions as described in \cite{rummery}. To extend this implementation to a continuous action space, our feed-forward neural network outputs discrete ``wires'' when given a state. Each wire consists of an action with its respective $Q$-value for the state given to the neural network. These wires may be interpolated to model $Q$, allowing us to get the $Q$-value of any action performed in state given as an input to the network. The interpolator used in Fido is a wire-fitted moving least squares interpolator used in the context of a memory-based learning system in \cite{baird}.

The $Q$-value of an action $\hat{a}$ for a state $s$ given a set of $n$ actions $a$ each with a respective $Q$-value $q$ is calculated as such:

\begin{equation}
	Q(a, s) = \cfrac{\sum_{i=0}^{n}\cfrac{q_i}{||\hat{a}-a_i||^2+c(q_{max}-q_i)+k}}{\sum_{i=0}^{n}\cfrac{1}{||\hat{a}-a_i||^2+c(q_{max}-q_i)+k}}
	\,,
\end{equation}

where $q_{max}$ is the greatest $Q$-value among the set of $Q$-values $q$, and $k$ is a small constant that avoids division by 0. $c$ is the smoothing factor. The greater the smoothing factor, the smoother the interpolated function.


Figure \ref{fig::wirefitexample} is an example of interpolation on a set of wires. The graph shows the value of one-dimensional actions plotted against their respective Q-value.

\begin{figure}[ht]
    \centering
    \includegraphics[height=5cm]{Figures/WireFit.png}
	\caption{Moving Least Squares Interpolator (adapted from Gaskett, Wettergreen, \& Zelinsky, 1999)}
    \label{fig::wirefitexample}
\end{figure}
