The most prevalent control system used in mobile robotics is a procedurally programmed expert system (Biggs \& MacDonald, 2003).  Such systems use conditional logic in order to emulate a desired behavior.  However, such systems are limited in numerous respects.  First, they can only perform the specific task for which they were programmed to accomplish; the entire software must be rewritten in order to change the target task.  Second, they rely on a knowledge of the inputs and outputs to the robot (such as sensors and motor control) in order to function.  The purpose of Fido was to solve both of these problems, allowing a universal general control system for robots that can be trained on tasks using reinforcement learning.

We chose to approach this problem with artificial neural networks; function appropriators modeled after nature with the capability to take in a large number of inputs to produce an output.  Neural networks are commonly used to solve tasks that are challenging using traditional rule-based programming, making them perfect for our task.  The control system was named Fido for the name's connotations to training an intelligent organism. 